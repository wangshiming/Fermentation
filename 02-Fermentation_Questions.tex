\documentclass[]{beamer}
\usetheme{Goettingen}
\usepackage{amsmath,siunitx,comment}
\begin{document}
\parskip=1cm plus 1pt
\newcounter{questions}
\section{Balances and Microbial Rates}

%%%%%%%%%%%%%%%%%%%%%%%%%%%%%%%%%%%%%%%%%%%%%%%%%%%%
\subsection{Microorganisms and Their Function in Nature}
\setcounter{questions}{0}
\begin{frame}[shrink] {}
\addtocounter{questions}{1}
\color{blue}
Where will you not find microorganisms? (\color{red}{Q\arabic{questions}})\\
\color{black}
\setlength{\parindent}{-0.4cm}
{\color{red}$\circ$} On the ocean floor \\
{\color{red}$\circ$} On acid waters  \\
{\color{red}$\circ$} In ice  \\
{\color{red}$\circ$} None of the above, microorganisms can be found in all these places. \\
\end{frame}

\begin{frame}[shrink] {}
\addtocounter{questions}{1}
\color{blue}
Why are microorganisms so important in the global cycling of elements? (\color{red}{Q\arabic{questions}})\\
\color{black}
\setlength{\parindent}{-0.4cm}
{\color{red}$\circ$} They break down organic material into e.g. $CO_2$ and $H_2O$, which can
then be converted into organic material again by photosynthesis;\\
{\color{red}$\circ$} When they die, they disintegrate into their constituting elements,
which fuel the element cycles;\\
{\color{red}$\circ$} The massive amounts of microorganisms present on earth
consume excess heat produced by the earth and sun in the process of
then elemental cycling. Without organisms the earth would therefore heat up.
\end{frame}

\begin{frame}[shrink] {}
\addtocounter{questions}{1}
\color{blue}
If you have one bacterial cell, dividing every hour, how many bacterial
cells would there be after 24 hours? (\color{red}{Q\arabic{questions}})\\
\color{black}
\setlength{\parindent}{-0.4cm}
{\color{red}$\circ$}\\ 
\end{frame}

\begin{frame}[shrink] {}
\addtocounter{questions}{1}
\color{blue}
Which effects can mutations in the DNA of microorganisms have? (\color{red}{Q\arabic{questions}})\\
\color{black}
\setlength{\parindent}{-0.4cm}
{\color{red}$\circ$} Any mutation in an organism's DNA is lethal. This will kill the cell
thereby ensuring that a cell’s DNA will stay the same;\\
{\color{red}$\circ$} DNA mutations have no effect on the organism, because there are
so little mutations compared to the size of the DNA molecule;\\
{\color{red}$\circ$} DNA mutations can be lethal, killing the organism; or silent, having
no relevant effect. But sometimes a mutation gives the organism a
competitive advantage over other microorganisms, favouring its
growth. These mutations are then carried on in future generations.
\end{frame}

\begin{frame}[shrink] {}
\addtocounter{questions}{1}
\color{blue}
Where in nature will you likely find microorganisms which degrade a
pesticide? (\color{red}{Q\arabic{questions}})\\
\color{black}
\setlength{\parindent}{-0.4cm}
{\color{red}$\circ$} in soil\\
{\color{red}$\circ$} in my garden\\
{\color{red}$\circ$} in soil close to the pesticide producing factory\\
{\color{red}$\circ$} close to volcanoes\\
{\color{red}$\circ$} organisms could never degrade a pesticide 
\end{frame}

\begin{frame}[shrink] {}
\addtocounter{questions}{1}
\color{blue}
Volcanoes produce a lot of gas containing $H_2S$. Which microorganisms do
you expect to be abundant around volcanoes? (\color{red}{Q\arabic{questions}})\\
\color{black}
\setlength{\parindent}{-0.4cm}
{\color{red}$\circ$} glucose consuming microorganisms\\
{\color{red}$\circ$} $H_2S$ consuming microorganisms which grow at $\SI{80}{\degreeCelsius}$\\
{\color{red}$\circ$} $H_2S$ consuming microorganisms which grow at $\SI{10}{\degreeCelsius}$. 
\end{frame}
\begin{frame}[shrink] {}
\addtocounter{questions}{1}
\color{blue}
What is the typical order of magnitude for the size of microorganisms? (\color{red}{Q\arabic{questions}})\\
\color{black}
\setlength{\parindent}{-0.4cm}
{\color{red}$\circ$}\\
\end{frame}
\begin{frame}[shrink] {}
\addtocounter{questions}{1}
\color{blue}
Where do you expect to find the largest variety of microorganisms? (\color{red}{Q\arabic{questions}})\\
\color{black}
\setlength{\parindent}{-0.4cm}
{\color{red}$\circ$} in air\\
{\color{red}$\circ$} in soil\\
{\color{red}$\circ$} in sea water\\
\end{frame}
\subsection{Functional understanding of nutrient requirement}
\setcounter{questions}{0}
\begin{frame}[shrink] {}
\addtocounter{questions}{1}
\color{blue}
Wet cells mostly consist of: (\color{red}{Q\arabic{questions}})\\
\color{black}
\setlength{\parindent}{-0.4cm}
{\color{red}$\circ$} Water\\ 
{\color{red}$\circ$} Lipids\\
{\color{red}$\circ$} Carbohydrates\\
{\color{red}$\circ$} DNA \\
\end{frame}
\begin{frame}[shrink] {}
\addtocounter{questions}{1}
\color{blue}
Which of the following compounds are required to produce biomass $C_1H_{1.80}O_{0.5}N_{0.2}$ (\color{red}{Q\arabic{questions}})\\
\color{black}
\setlength{\parindent}{-0.4cm}
{\color{red}$\circ$}  Carbon source\\
{\color{red}$\circ$} Nitrogen source\\
{\color{red}$\circ$} Trace metals\\
{\color{red}$\circ$} All of the above \\
\end{frame}

\begin{frame}[shrink] {}
\addtocounter{questions}{1}
\color{blue}
An example of an inorganic compound required for microbial growth is:  (\color{red}{Q\arabic{questions}})\\
\color{black}
\setlength{\parindent}{-0.4cm}
{\color{red}$\circ$}  Glucose\\
{\color{red}$\circ$} Vitamins\\
{\color{red}$\circ$} Lipids\\
{\color{red}$\circ$} Potassium \\
\end{frame}

\begin{frame}[shrink] {}
\addtocounter{questions}{1}
\color{blue}
Where do microorganisms get the energy to ‘drive’ the synthesis of biomass (anabolism)?  (\color{red}{Q\arabic{questions}})\\
\color{black}
\setlength{\parindent}{-0.4cm}
{\color{red}$\circ$}  No need for that, synthesis of biomass is a spontaneous reaction;\\
{\color{red}$\circ$} By using the energy released when an electron donor is oxidized and an electron acceptor is reduced, both to compounds with a lower energetic state (catabolism);\\
{\color{red}$\circ$} From heat in their environment;\\
{\color{red}$\circ$} None of the above. \\
\end{frame}

\begin{frame}[shrink] {}
\addtocounter{questions}{1}
\color{blue}
Dried biomass mostly contains:  (\color{red}{Q\arabic{questions}})\\
\color{black}
\setlength{\parindent}{-0.4cm}
{\color{red}$\circ$} Lipids\\
{\color{red}$\circ$} RNA\\
{\color{red}$\circ$}  Protein \\
{\color{red}$\circ$} Carbohydrate\\
{\color{red}$\circ$} Water \\
\end{frame}

\begin{frame}[shrink] {}
\addtocounter{questions}{1}
\color{blue}
A microorganism grows on methanol and uses $NO_{3}N^{-}$ as N-source. Which compounds does the anabolic reaction for biomass formation contain?   (\color{red}{Q\arabic{questions}})\\
\color{black}
\setlength{\parindent}{-0.4cm}
{\color{red}$\circ$}  Methanol\\
{\color{red}$\circ$} $NO_{3}^{-}$\\
{\color{red}$\circ$} ${H_{2}O}$\\
{\color{red}$\circ$} Biomass\\
{\color{red}$\circ$} $HCO_{3}^{-}$ (or $CO_2$)\\
{\color{red}$\circ$} $O_2$\\
{\color{red}$\circ$} Glucose \\
\end{frame}

\begin{frame}[shrink] {}
\addtocounter{questions}{1}
\color{blue}
Which compounds can function as N-source?   (\color{red}{Q\arabic{questions}})\\
\color{black}
\setlength{\parindent}{-0.4cm}
{\color{red}$\circ$} $NH_{4}^{+}$ \\
{\color{red}$\circ$} $NO_{3}^{-}$ \\
{\color{red}$\circ$} Urea\\
{\color{red}$\circ$} Amino acids\\
\end{frame}

\begin{frame}[shrink] {}
\addtocounter{questions}{1}
\color{blue}
Which of the following nutrients can be synthesized by microorganisms?   (\color{red}{Q\arabic{questions}})\\
\color{black}
\setlength{\parindent}{-0.4cm}
{\color{red}$\circ$}  Trace metals\\
{\color{red}$\circ$} Vitamins\\
{\color{red}$\circ$} Lipids \\
\end{frame}

\begin{frame}[shrink] {}
\addtocounter{questions}{1}
\color{blue}
Lipids are present in:   (\color{red}{Q\arabic{questions}})\\
\color{black}
\setlength{\parindent}{-0.4cm}
{\color{red}$\circ$} DNA\\
{\color{red}$\circ$} RNA\\
{\color{red}$\circ$} All membranes\\
{\color{red}$\circ$} Ribosome \\
\end{frame}

\begin{frame}[shrink] {}
\addtocounter{questions}{1}
\color{blue}
Which of the following compounds can be an electron acceptor?   (\color{red}{Q\arabic{questions}})\\
\color{black}
\setlength{\parindent}{-0.4cm}
{\color{red}$\circ$}  ${NO_{3}}^{-}$\\
{\color{red}$\circ$} ${NO_{2}}^{-}$\\
{\color{red}$\circ$} ${SO_{4}}^{2-}$\\
{\color{red}$\circ$} $S_0$\\
{\color{red}$\circ$} $H_{2}S$ \\
\end{frame}

\begin{frame}[shrink] {}
\addtocounter{questions}{1}
\color{blue}
Which of the following compounds can be an electron donor?   (\color{red}{Q\arabic{questions}})\\
\color{black}
\setlength{\parindent}{-0.4cm}
{\color{red}$\circ$}  ${NO_{3}}^{-}$\\
{\color{red}$\circ$} ${NO_{2}}^{-}$\\
{\color{red}$\circ$} ${SO_{4}}^{2-}$\\
{\color{red}$\circ$} $S_0$\\
{\color{red}$\circ$} $H_{2}S$ \\
\end{frame}

\subsection{Broth balances}
\setcounter{questions}{0}

\begin{frame}[shrink] {}
\addtocounter{questions}{1}
\color{blue}
Which balance must be set up to calculate the rate of product production “$R_p$” from experimental measurement in a fermenter? (\color{red}{Q\arabic{questions}})\\
\color{black}
\setlength{\parindent}{-0.4cm}
{\color{red}$\circ$} Biomass balance\\
{\color{red}$\circ$} Substrate balance\\
{\color{red}$\circ$} Product Balance \\
\end{frame}

\begin{frame}[shrink] {}
\addtocounter{questions}{1}
\color{blue}
Which measurements are needed to calculate the production rate $R_p$ in a continuous process of a non-volatile stable product ? (\color{red}{Q\arabic{questions}})\\
\color{black}
\setlength{\parindent}{-0.4cm}
{\color{red}$\circ$}  $F_{in},F_{out},c_{s},V_{L},time$  \\
{\color{red}$\circ$}  $F_{in},F_{out},c_{p,in},c_{p,out},c_p$ \\
{\color{red}$\circ$}  $F_{in},F_{out},c_{p,in},c_{p,out},time$ \\
{\color{red}$\circ$}  $F_{in},F_{out},c_{p,in},c_{p,out}$\\
\end{frame}

\begin{frame}[shrink] {}
\addtocounter{questions}{1}
\color{blue}
If you know that no product enters a fermenter in the inflow, no product is consumed (e.g. in a degradation reaction), and the product is not volatile, the compound balance for this product can be formulated as:(\color{red}{Q\arabic{questions}})\\
\color{black}
\setlength{\parindent}{-0.4cm}
{\color{red}$\circ$}  Accumulation = in - out + production - consumption\\
{\color{red}$\circ$} Accumulation = in - out + production \\
{\color{red}$\circ$} Accumulation = - out + production \\
{\color{red}$\circ$} Accumulation = In - out - consumption
\end{frame}

\begin{frame}[shrink] {}
\addtocounter{questions}{1}
\color{blue}
What’s the unit of $R_p$ in the product balance when concentrations are given in mol/L?(\color{red}{Q\arabic{questions}})\\
\color{black}
\setlength{\parindent}{-0.4cm}
{\color{red}$\circ$} mol product/h \\
{\color{red}$\circ$} kg product/h\\
{\color{red}$\circ$} kg/h\\
{\color{red}$\circ$} mol substrate/h\\
\end{frame}

\begin{frame}[shrink] {}
\addtocounter{questions}{1}
\color{blue}
In an experiment a sample is taken from the broth outflow in order to determine the product concentration. It is known that the product is secreted from the cells by active transport, which means that the intracellular product concentration is negligible. The following steps were performed when taking the sample:\\
{\color{red}$\triangleright$} Sampling of the broth\\
{\color{red}$\triangleright$} Removal of biomass by filtration to obtain the broth filtrate\\
{\color{red}$\triangleright$} Analysis of product concentration in the broth filtrate in mol P/m3 broth filtrate\\[0.5em]

  Give the unit of $F_{out}$ in the product balance which is needed to calculate $R_p$ (in mol P/h)(\color{red}{Q\arabic{questions}})\\
\color{black}
\setlength{\parindent}{-0.4cm}
{\color{red}$\circ$}  kg broth/h\\
{\color{red}$\circ$} $m^3$ broth filtrate/h \\
{\color{red}$\circ$} $m^3$ total broth/h \\
\end{frame}

\begin{frame}[shrink] {}
\addtocounter{questions}{1}
\color{blue}
$R_p$ is calculated by the product balance, only production, medium inflow and broth outflow are taken into account.
When other mechanisms (degradation, evaporation) apply to the product, then the calculated. $R_p$ (\color{red}{Q\arabic{questions}})\\
\color{black}
\setlength{\parindent}{-0.4cm}
{\color{red}$\circ$} Is correct\\
{\color{red}$\circ$} Is too low \\
{\color{red}$\circ$} Is too high \\
\end{frame}

\begin{frame}[shrink] {}
\addtocounter{questions}{1}
\color{blue}
In a continuous process Rp must be calculated in mol product/h. The molar mass of the product is 110 g/mol. The following measurements are available:({\color{red}{Q\arabic{questions}}})\\[0.3em]
\color{gray}
\begin{tabular}[]{l l l}
Measurement & Value & Unit\\
Fin & 1.00 & L/h\\
cp,in& 	100& 	g/m3\\
Fout& 	1.2& 	L/h\\
cp,out&  	10 & 	mol/m3\\
cp (fermenter) & 11 & 	mol/m3\\
\end{tabular}\\[0.3em]
\color{black}
\setlength{\parindent}{-0.4cm}
{\color{red}$\circ$} \\
 
\end{frame}

\begin{frame}[shrink] {}
\addtocounter{questions}{1}
\color{blue}
Which balance is needed to obtain $R_s$? (\color{red}{Q\arabic{questions}})\\[0.3em]
\color{black}
\setlength{\parindent}{-0.4cm}
{\color{red}$\circ$}  Product balance  \\
{\color{red}$\circ$} Consumption balance\\
{\color{red}$\circ$} Production balance \\
{\color{red}$\circ$} Substrate balance 
\end{frame}

\begin{frame}[shrink] {}
\addtocounter{questions}{1}
\color{blue}
Which measurements are needed to obtain $R_s$ in a steady state continuous process?  (\color{red}{Q\arabic{questions}})\\
\color{black}
\setlength{\parindent}{-0.4cm}
{\color{red}$\circ$}  $c_{s,in}$\\
{\color{red}$\circ$} $V_L$\\
{\color{red}$\circ$} $F_{in}$\\
{\color{red}$\circ$} $F_{out}$\\
{\color{red}$\circ$} $N_s$\\
{\color{red}$\circ$} $c_s$
\end{frame}

\begin{frame}[shrink] {}
\addtocounter{questions}{1}
\color{blue}
Which assumptions have been made in setting up the balance you would use to calculate $R_s$ (\color{red}{Q\arabic{questions}})\\
\color{black}
\setlength{\parindent}{-0.4cm}
{\color{red}$\circ$}   All the substrate is consumed  \\
{\color{red}$\circ$} No degradation of the substrate   \\
{\color{red}$\circ$} The substrate is non-volatile   \\
{\color{red}$\circ$} No substrate is added to the feed  \\
{\color{red}$\circ$} Only a portion of the substrate is consumed
\end{frame}

\subsection{Gas phase balances}
\setcounter{questions}{0}

\begin{frame}[shrink] {}
\addtocounter{questions}{1}
\color{blue}
What could be the units of $T_{N,O}$ in the gas phase $O_2$-balance for a fermenter? (\color{red}{Q\arabic{questions}})\\
\color{black}
\setlength{\parindent}{-0.4cm}
{\color{red}$\circ$} Kg O2/h  \\
{\color{red}$\circ$} Kg/h  \\
{\color{red}$\circ$} mol H2O/h\\
{\color{red}$\circ$} mol N2/h  \\
{\color{red}$\circ$} mol CO2/h \\
{\color{red}$\circ$} mol CO2/(m3 broth h) \\
{\color{red}$\circ$} mol O2/h
\end{frame}

\begin{frame}[shrink] {}
\addtocounter{questions}{1}
\color{blue}
Which of the following terms are present in a gas phase O2-balance?  (\color{red}{Q\arabic{questions}})\\
\color{black}
\setlength{\parindent}{-0.4cm}
{\color{red}$\circ$}   $F_{N}y_{O,in}$\\   
{\color{red}$\circ$}   $F_{N,out}y_{O,out}$\\   
{\color{red}$\circ$} $T_{N,O}$\\
{\color{red}$\circ$} $R_O$\\
{\color{red}$\circ$}   $F_{N,in}y_{O,in}$  \\ 
{\color{red}$\circ$}   $F_{N,out}y_{O,in}$   
\end{frame}

\begin{frame}[shrink] {}
\addtocounter{questions}{1}
\color{blue}
In which combination of the following situations are the molar gas in- and outflows equal?  (\color{red}{Q\arabic{questions}})\\
\color{black}
\setlength{\parindent}{-0.4cm}
{\color{red}$\circ$}   No water evaporation\\
{\color{red}$\circ$} $T_{N,C}$ and $T_{N,O}$ are equal and opposite  \\
{\color{red}$\circ$} No product and substrate evaporation\\
{\color{red}$\circ$} Equal temperature and pressure for gas in and outflow    \\
\end{frame}

\begin{frame}[shrink] {}
\addtocounter{questions}{1}
\color{blue}
You have used the biomass product and substrate balances in broth to calculate RX, RP and RS. Product and biomass both contain N which is available as (NH4)2SO4 in the feed solution. Also pH is low enough that NH3 does not evaporate from broth.\\[0.3em]
Which additional balance(s) (and their measurements) is/are needed to apply N-conservation? (\color{red}{Q\arabic{questions}})\\
\color{black}
\setlength{\parindent}{-0.4cm}
{\color{red}$\circ$}   a gas phase O2-balance \\
{\color{red}$\circ$} a gas phase CO2-balance \\
{\color{red}$\circ$} a gas phase NH2-balance \\
{\color{red}$\circ$} a broth ${NH_4}^+$-balance  \\
\end{frame}

\begin{frame}[shrink] {}
\addtocounter{questions}{1}
\color{blue}
Which balance(s) is/are needed to calculate $F_{N,out}$ (in mol gas/h)?  (\color{red}{Q\arabic{questions}})\\
\color{black}
\setlength{\parindent}{-0.4cm}
{\color{red}$\circ$}    The broth substrate balance  \\
{\color{red}$\circ$} The gas phase O2-balance  \\
{\color{red}$\circ$} The gas phase N2-balance\\
{\color{red}$\circ$} The gas phase CO2-balance  \\
\end{frame}

\begin{frame}[shrink] {}
\addtocounter{questions}{1}
\color{blue}
Which of the following terms are present in a gas phase O2-balance?  (\color{red}{Q\arabic{questions}})\\
\color{black}
\setlength{\parindent}{-0.4cm}
{\color{red}$\circ$}   $F_{N}y_{O,in}$\\   
{\color{red}$\circ$}   $F_{N,out}y_{O,out}$\\   
{\color{red}$\circ$} $T_{N,O}$\\
{\color{red}$\circ$} $R_O$\\
{\color{red}$\circ$}   $F_{N,in}y_{O,in}$  \\ 
{\color{red}$\circ$}   $F_{N,out}y_{O,in}$   
\end{frame}

\begin{frame}[shrink] {}
\addtocounter{questions}{1}
\color{blue}
What could be the reason your carbon balances do not close?  (\color{red}{Q\arabic{questions}})\\
\color{black}
\setlength{\parindent}{-0.4cm}
{\color{red}$\circ$}    Measurement errors\\
{\color{red}$\circ$} Unknown substrate consuming processes\\
{\color{red}$\circ$} Neither A nor B \\
{\color{red}$\circ$} Both A and B   \\
\end{frame}

\begin{frame}[shrink] {}
\addtocounter{questions}{1}
\color{blue}
Why is it practically impossible to use the broth water balance to obtain accurate values of RW (mol H2O/h produced or consumed) in fermentation processes?  (\color{red}{Q\arabic{questions}})\\
\color{black}
\setlength{\parindent}{-0.4cm}
{\color{red}$\circ$}    There is water evaporation\\
{\color{red}$\circ$} There is no net water production or consumption  \\
{\color{red}$\circ$} Because part of the water constantly switches between the gas phase and the water phase when a condensor is used \\
{\color{red}$\circ$} $R_w$ is very small compared to water in- and outflow in a continuous process   
\end{frame}

\begin{frame}[shrink] {}
\addtocounter{questions}{1}
\color{blue} For sterility reasons the air sparged into the fermenter is dry. The gas which leaves the fermenter is saturated with water. Measurements of O2 and CO2 mole fractions in the off-gas (gas that leaves the fermenter) therefore require a special treatment of the gas before it can be measured. Which treatment is this?  (\color{red}{Q\arabic{questions}})\\
\color{black}
\setlength{\parindent}{-0.4cm}
{\color{red}$\circ$}     Drying of the off-gas \\
{\color{red}$\circ$} Heating of the off-gas \\
{\color{red}$\circ$} Sterile filtration of the off-gas\\
{\color{red}$\circ$} No treatment is required  
\end{frame}

\begin{frame}[shrink] {}
\addtocounter{questions}{1}
\color{blue}
Previously we used C-conservation as a check on the obtained values of $R_x$, $R_S$ and $R_P$.  Which other conservation relations can we apply?  (\color{red}{Q\arabic{questions}})\\
\color{black}
\setlength{\parindent}{-0.4cm}
{\color{red}$\circ$}     O-conservation \\
{\color{red}$\circ$} H-conservation\\
{\color{red}$\circ$} N-conservation  \\
\end{frame}
\subsection{The Microorganism's $q$ rates and chemostat} 
\setcounter{questions}{0}

\begin{frame}[shrink] {}
\addtocounter{questions}{1}
\color{blue}
 Which is the correct general equation to obtain $\mu$ from measurements? (\color{red}{Q\arabic{questions}})\\
\color{black}
\setlength{\parindent}{-0.4cm}
{\color{red}$\circ$} $\mu= D = \frac{F}{V}$ \\ 
{\color{red}$\circ$} $\mu = \frac{\mathrm d}{\mathrm d x}c_X$\\
{\color{red}$\circ$} $\mu = \frac{R_X}{V_{L}c_X}$ \\
{\color{red}$\circ$} $\mu = \frac{\mathrm dc_X}{\mathrm d x} \frac{1}{c_X}$\\
\end{frame}

\begin{frame}[shrink] {}
\addtocounter{questions}{1}
\color{blue}
Which of the following changes leads to a change in the steady state $\mu$ value in a chemostat which is always operated at Fout=0.10 $m^{3}/h$ and $V_L$=1.00$m^3$? (\color{red}{Q\arabic{questions}})\\
\color{black}
\setlength{\parindent}{-0.4cm}
{\color{red}$\circ$} Change in T\\
{\color{red}$\circ$} Change in pH\\
{\color{red}$\circ$} Change in $C_{s,in}$ \\
{\color{red}$\circ$} Change in product concentration\\
{\color{red}$\circ$} When $C_{x,out}$ becomes different from $C_{x,fermenter}$ \\
{\color{red}$\circ$} None will change $\mu$\\
\end{frame}

\begin{frame}[shrink] {}
\addtocounter{questions}{1}
\color{blue}
Which of the following $q_s$-rates are incorrect? (\color{red}{Q\arabic{questions}})\\
\color{black}
\setlength{\parindent}{-0.4cm}
{\color{red}$\circ$} $q_s = kgS*h^{-1}/molX$ \\ 
{\color{red}$\circ$} $q_s = molS*h^{-1}/molX$ \\ 
{\color{red}$\circ$} $q_s = molS*h^{-1}/m^{3}broth$ \\ 
{\color{red}$\circ$} $q_s = kgS*s^{-1}/molS$ \\ 
{\color{red}$\circ$} $q_s = kgS*s^{-1}/m^{3}broth$ \\ 
\end{frame}

\begin{frame}[shrink] {}
\addtocounter{questions}{1}
\color{blue}
How can we compare the performance of different microorganisms? You can assume that the product is secreted. (\color{red}{Q\arabic{questions}})\\
\color{black}
\setlength{\parindent}{-0.4cm}
{\color{red}$\circ$}  By looking at them under a microscope\\
{\color{red}$\circ$} By comparing the calculated q-rates By comparing the calculated q-rates \\
{\color{red}$\circ$} By comparing the product concentrations in the fermentor \\
\end{frame}

\begin{frame}[shrink] {}
\addtocounter{questions}{1}
\color{blue}
How can we test whether ideal broth outflow in a chemostat is realized? (\color{red}{Q\arabic{questions}})\\
\color{black}
\setlength{\parindent}{-0.4cm}
{\color{red}$\circ$}  By measuring $C_{x,fermenter}$ at increasing stirrer speed\\
{\color{red}$\circ$} By measuring biomass concentrations in the broth inside the fermenter and broth in the outflow \\
\end{frame}

\begin{frame}[shrink] {}
\addtocounter{questions}{1}
\color{blue}
 What are important characteristics of a steady state process? (\color{red}{Q\arabic{questions}})\\
\color{black}
\setlength{\parindent}{-0.4cm}
{\color{red}$\circ$}   The concentrations and in- and outflow rates are constant in time \\
{\color{red}$\circ$} The product isn’t produced too fast\\
{\color{red}$\circ$} The microorganisms aren’t growing anymore, just producing product \\
\end{frame}

\begin{frame}[shrink] {}
\addtocounter{questions}{1}
\color{blue}
Which assumptions are important to verify when setting up component balances for a chemostat? (\color{red}{Q\arabic{questions}})\\
\color{black}
\setlength{\parindent}{-0.4cm}
{\color{red}$\circ$} Whether the assumption of ideal mixing is valid\\
{\color{red}$\circ$} Whether the inflow Fin equals the outflow Fout\\
{\color{red}$\circ$} Whether $C_X = C_{X,out}$ and $P_X = C_{P,out}$ \\
{\color{red}$\circ$} Whether there is no evaporation or depletion of substrate, product and biomass.\\
{\color{red}$\circ$} Whether the volume is constant  \\
\end{frame}

\begin{frame}[shrink] {}
\addtocounter{questions}{1}
\color{blue}
In a chemostat with ideal broth outflow an organism makes a product. The following measurements are available: (\color{red}{Q\arabic{questions}})\\
\color{gray}
 \begin{tabular}[ ]{l l l}
Measurement & Value & Unit \\
$V_L$ & 5.00 & $m^3$ \\
$F_{in}$ & 0.1 & $m^{3}/h$ \\
$F_{out}$ & 0.12 & $m^{3}/h$ \\
$C_{X}$ & 1000 & $mol/h$ \\
$C_{P}$ & 1000 & $mol/h$ \\
 \end{tabular} \\
\color{black}
In the above mentioned chemostat $c_{S,out} = c_{S} =20mol/m^3$ and $c_{S,in} = 2000 mol/m^3$. \\[0.3em]
Calculate $\mu, q_P, q_S$ and give the correct units. Give answers in four digits after the dot. \\
\setlength{\parindent}{-0.4cm}
{\color{red}$\circ$} $\mu=$\\ 
{\color{red}$\circ$} $q_P=$ \\ 
{\color{red}$\circ$} $q_S=$ \\
\end{frame}

\begin{frame}[shrink] {}
  A second experiment is performed to test if product inhibition affects this fermentation. In this experiment, product was added to the feed ($C_{P,in} = 1000 mol/m^3$) to increase the product concentration in the fermenter. It was found that $C_{P,out2} =C_p  1833 mol P/m^3$ in this experiment. All the other values have stayed the same as in the previous fermentation. \\[0.3em]
  Calculate the $q_{P2}$: \\ [0.3em]
{\color{red}$\circ$} $q_{P2}=$\\[0.5em] 
Compare the value of $q_{P2}$ with the value of $q_{P}$ found in question 8A. Based on these results, is there product inhibition? \\[0.3em]
{\color{red}$\circ$} \\ 
\end{frame}

%%%%%%%%%%%%%%%%%%%%%%%%%%%%%%%%%%%%%%%%%%%%%%%%%%%%%

\subsection{Bath mode operation}
\setcounter{questions}{0}
\begin{frame}[shrink] {}
\addtocounter{questions}{1}
\color{blue}
Which phenomena in a batch fermentation contribute to an increasing or decreasing broth mass in time? (\color{red}{Q\arabic{questions}})\\
\color{black}
\setlength{\parindent}{-0.4cm}
{\color{red}$\circ$}  Biomass growth \\
{\color{red}$\circ$} Water evaporation \\
{\color{red}$\circ$} Titrant addition \\
{\color{red}$\circ$} Product evaporation \\
{\color{red}$\circ$} Temperature control  \\
\end{frame}

\begin{frame}[shrink] {}
\addtocounter{questions}{1}
\color{blue}
Which terms do not occur in the broth O2 – balance in a batch fermentation? (\color{red}{Q\arabic{questions}})\\
\color{black}
\setlength{\parindent}{-0.4cm}
{\color{red}$\circ$} $\frac{\mathrm d [V_{L}(t)C_O(t)]}{\mathrm dt}$ \\[0.3em]
{\color{red}$\circ$} $F_{N,out}(t) y_{O,out}(t)$ \\
{\color{red}$\circ$} $T_{N,O} (t)$ \\
{\color{red}$\circ$} $R_O (t)$ \\
{\color{red}$\circ$} $F_{N,in}(t) y_{O,in}(t)$ \\
{\color{red}$\circ$} $F_{in}(t) C_{O,in}(t)$ \\
{\color{red}$\circ$} $F_{out}(t) C_{O,out}(t)$ \\
\end{frame}

\begin{frame}[shrink] {}
\addtocounter{questions}{1}
\color{blue}
In a batch fermentation the following measurements were done: \\
\color{gray}
$V_L$ = 1.5 L broth  \\
$C_X$ = 100 g dry matter/L = 300 g wet biomass/L \\
$C_P$ = 20 g/L filtrate \\
\color{blue}
How much gram product is present in the broth of the whole batch? 
(Assume that the density of biomass is 1 kg/L) (\color{red}{Q\arabic{questions}})\\
\color{black}
{\color{red}$\circ$}  \\
\end{frame}

\begin{frame}[shrink] {}
\addtocounter{questions}{1}
\color{blue}
In question 3 an essential assumption about the product was made: (\color{red}{Q\arabic{questions}})\\
\color{black}
\setlength{\parindent}{-0.4cm}
{\color{red}$\circ$}  The product is stable \\
{\color{red}$\circ$} The product is volatile \\
{\color{red}$\circ$} The product is present in the organism at a negligible concentration.  \\
\end{frame}

\begin{frame}[shrink] {}
\addtocounter{questions}{1}
\color{blue}
Broth is: (\color{red}{Q\arabic{questions}})\\
\color{black}
\setlength{\parindent}{-0.4cm}
{\color{red}$\circ$}  Water with nutrients \\
{\color{red}$\circ$} Water with products \\
{\color{red}$\circ$}  Water with nutrients, product, biomass and by-products \\
{\color{red}$\circ$} Water with biomass  \\
\end{frame}

\begin{frame}[shrink] {} 
\color{blue}
    
The batch balance of a volatile substrate has the following terms: ({\color{red}{Q6}})\\[0.5em]
\color{black}
\setlength{\parindent}{-0.4cm}
{\color{red}$\circ$} $\frac{d[V_L(t)C_S(t)]}{dt}$  \\
{\color{red}$\circ$} $q_{S}$$V_{L}$(t)$c_{S}$(t)  \\
{\color{red}$\circ$} $T_{N,S}$(t)  \\
{\color{red}$\circ$} $R_{S}$(t)  \\
{\color{red}$\circ$} $F_{in}$, $c_{S,in}$  \\
{\color{red}$\circ$} $q_{S}$$V_{L}$(t)$c_{X}$(t)  \\
{\color{red}$\circ$} $F_{in}$$C_{s,out}$  \\
\end{frame}


\begin{frame}[shrink] {} 
\color{blue}
  In a batch experiment in the exponential growth phase half of the broth is removed, the temperature, pH and airflow are kept constant.
 ({\color{red}{Q7}})\\
  Which of the following items change due to the broth removal?
  \\
\color{black}
\setlength{\parindent}{-0.4cm}
{\color{red}$\circ$} $\mu$  \\
{\color{red}$\circ$} $q_{S}$  \\
{\color{red}$\circ$} $R_{X}$  \\
{\color{red}$\circ$} $R_{S}$  \\
{\color{red}$\circ$} $c_{X}$  \\
{\color{red}$\circ$} $y_{CO2,out}$  \\
\end{frame}


\begin{frame}[shrink] {} 
\color{blue}
  In a batch experiment in the following measurements have been done:
 ({\color{red}{Q8}})\\
\color{gray}
\begin{tabular}[ ]{l l l}
Time & $C_{X}$ (in g dry matter/L) & $V_{L}$ (in L)  \\
3.00 & 20 & 10.0  \\
4.00 & 25 & 8.0  \\
\end{tabular}  \\
\color{blue}
  Calculate $\mu$ in the time interval 3 to 4 hours:
  \\
\color{black}
\setlength{\parindent}{-0.4cm}
{\color{red}$\circ$} $0 ~h^{-1}$  \\
{\color{red}$\circ$} $0.233 ~h^{-1}$  \\
{\color{red}$\circ$} $1.25 ~h^{-1}$  \\
\end{frame}


\begin{frame}[shrink] {} 
\color{blue}
In a batch experiment one adds glucose at t=0 with an amount $N_{S}$(0)= 110 mol glucose. One also inoculates with biomass at t=0, with biomass amount $N_{X}$(0)=20 mol X. ({\color{red}{Q9}})\\
At the end, after 15 hours, $N_{X}$(15) = 320 mol X and $N_{S}$(15) = 10 mol glucose.  \\
Calculate $\mu$$_{max}$ and q$_{S,max}$.  \\[.5em]
\color{black}
\setlength{\parindent}{-0.4cm}
{\color{red}$\circ$} $\mu$$_{max}$:\quad  \\
{\color{red}$\circ$} $q_{S,max}$:\quad \\
\end{frame}


\begin{frame}[shrink] {} 
\color{blue}
To obtain $\mu$$_{max}$ from a proportional plot (note: a proportional plot goes through the origin (0,0) of a x,y plot) in a batch fermentation one must plot as function of time: ({\color{red}{Q10}})\\[0.5em]
\color{black}
\setlength{\parindent}{-0.4cm}
{\color{red}$\circ$} $c_{X}$(t)  \\[0.3em]
{\color{red}$\circ$} $V_{L}$(t)$c_{X}$(t)  \\[0.3em]
{\color{red}$\circ$} $ln(C_x(t))$  \\[0.3em]
{\color{red}$\circ$} $ln(N_X(t))$  \\[0.3em]
{\color{red}$\circ$} $ln(\frac{N_X(t)}{N_X(0)})$  \\[0.3em]
\end{frame}


\begin{frame}[shrink] {} 
\color{blue}
    
 To obtain $q_{s,max}$ from a batch experiment with known $\mu$$_{max}$ from a proportional plot (note: a proportional plot goes through the origin (0,0) of a x,y plot) one must plot :  ({\color{red}{Q11}})\\[0.5em]
\color{black}
\setlength{\parindent}{-0.4cm}
{\color{red}$\circ$} $\frac{C_S(t)}{C_X(t)}$  \\[0.3em]
{\color{red}$\circ$} $\frac{C_S(t)-C_S(0)}{C_X(t)}$  \\[0.3em]
{\color{red}$\circ$} $\frac{N_S(t)-N_S(0)}{N_X(t)}$  \\[0.3em]
{\color{red}$\circ$} $\frac{N_S(t)-N_S(0)}{N_X(t)-N_X(0)}$  \\[0.3em]
\end{frame}


\section{The Black Box Model and Process Reaction}

\end{document}
