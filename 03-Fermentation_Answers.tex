\documentclass[]{beamer}
\usetheme{Goettingen}
\usepackage{amsmath,siunitx,comment}
\begin{document}
\parskip=1cm plus 1pt
\section{Biotechnology for Biobased Products}
\section{Balances and Microbial Rates}
\section{The Black Box Model and Process Reaction}

\subsection{The process reaction}
\begin{frame}[shrink] {} 
\color{blue}
  Which four of the following statements about the process reaction are correct?
 ({\color{green}{Q1}})\\
\color{black}
\setlength{\parindent}{-0.4cm}
{\color{red}$\circ$} The process reaction has constant stoichiometry for different $\mu$-values  \\
{\color{red}$\bullet$} The stoichiometric coefficient for a product is always positive  \\
{\color{red}$\bullet$} The process reaction always contains ($CO_{2}$), $H_{2}$O, $H^{+}$, and N-source as components  \\
{\color{red}$\circ$} The process reaction does not contain biomass  \\
{\color{red}$\bullet$} The stoichiometric coefficient of substrate is $q_{s}$/$q_{p}$  \\
{\color{red}$\circ$} The stoichiometric coefficient of substrate is -$q_{s}$/$q_{p}$  \\
{\color{red}$\bullet$} Stoichiometric coefficient of $O_{2}$ is $q_{O2}$/$q_{p}$  \\
\end{frame}


\begin{frame}[shrink] {} 
\color{blue}
  Which four of the following statements about the theoretical product reaction are correct?
 ({\color{green}{Q2}})\\
\color{black}
\setlength{\parindent}{-0.4cm}
{\color{red}$\circ$} The theoretical product reaction contains $O_{2}$ as component  \\
{\color{red}$\bullet$} The theoretical product reaction contains $CO_{2}$ as component  \\
{\color{red}$\bullet$} The stoichiometric coefficients of the theoretical product reaction are constant  \\
{\color{red}$\circ$} The theoretical product reaction contains biomass as component  \\
{\color{red}$\bullet$} When the composition of substrate and product are known one can calculate the complete theoretical product reaction  \\
{\color{red}$\circ$} The theoretical product reaction contains 1 mol substrate  \\
{\color{red}$\bullet$} The theoretical product reaction contains 1 mol product  \\
\end{frame}


\begin{frame}[shrink] {} 
\color{blue}
How many element conservation relations apply to the following theoretical product reaction of butanediol ($C_{4}$$H_{10}$$O_{2}$)? ({\color{green}{Q3a}})\\
\color{gray}
a $C_{6}$$H_{12}$$O_{6}$ + b C$O_{2}$ + c $H_{2}$O + 1 $C_{4}$$H_{10}$$O_{2}$  \\
\color{black}
\setlength{\parindent}{-0.4cm}
{\color{red}$\circ$} 4  \\
{\color{red}$\circ$} 2  \\
{\color{red}$\bullet$} 3  \\
{\color{red}$\circ$} 5  \\
\end{frame}


\begin{frame}[shrink] {} 
\color{blue}
Calculate the stoichiometric coefficients a, b and c in the theoretical Butanediol ($C_{4}$$H_{10}$$O_{2}$) product reaction from glucose ($C_{6}$$H_{12}$$O_{6}$) \\[0.4em]
\color{gray}
a $C_{6}$$H_{12}$$O_{6}$ + b C$O_{2}$ + c $H_{2}$O + 1 $C_{4}$$H_{10}$$O_{2}$ \\[0.4em]
\color{blue}
For each answer, indicate with at “+” or “-” whether the compound is produced or consumed (e.g. +0.18). Give your answer correct to three decimal places. ({\color{green}{Q3b}})\\[1em]
\color{black}
\setlength{\parindent}{-0.4cm}
{\color{red}$\bullet$} a = -0.917\\
{\color{red}$\bullet$} b = 1.500\\
{\color{red}$\bullet$} c = 0.500\\

\end{frame}


\begin{frame}[shrink] {} 
\color{blue}
How much gram glucose is required to produce 1 gram of butanediol in the theoretical product reaction? Give your answer correct to three decimal places. 
 ({\color{green}{Q3c}})\\
Use the following information.
\\[0.5em]
\color{gray}
Molar mass glucose = 180.1559 g/mol\\
Molar mass butanediol = 90.121 g/mol\\
\color{blue}
The theoretical product reaction:\\
\color{gray}
a $C_{6}$$H_{12}$$O_{6}$ + b C$O_{2}$ + c $H_{2}$O + 1 $C_{4}$$H_{10}$$O_{2}$  \\[1em]
\color{black}
\setlength{\parindent}{-0.4cm}
{\color{red}$\bullet$} Glucose (gram) = 1.832\\
\end{frame}


\begin{frame}[shrink] {} 
\color{blue}
  What is the unit of the coefficient a in the theoretical product reaction for Butanediol (see question 3A and 3B?)
 ({\color{green}{Q3d}})\\
\color{black}
\setlength{\parindent}{-0.4cm}
{\color{red}$\bullet$} Mol glucose/mol butanediol  \\
{\color{red}$\circ$} Mol butanediol/mol glucose  \\
\end{frame}


\begin{frame}[shrink] {} 
\color{blue}
  What is the unit of coefficient b in the theoretical product reaction for butanediol?
 ({\color{green}{Q3e}})\\
\color{black}
\setlength{\parindent}{-0.4cm}
{\color{red}$\circ$} Mol $CO_{2}$/mol glucose  \\
{\color{red}$\bullet$} Mol $CO_{2}$/mol Butanediol  \\
{\color{red}$\circ$} Mol butanediol/mol $CO_{2}$  \\
\end{frame}


\begin{frame}[shrink] {} 
\color{blue}
    Indicate which component is not always present in the process reaction?
 ({\color{green}{Q4a}})\\
\color{black}
\setlength{\parindent}{-0.4cm}
{\color{red}$\circ$} Substrate  \\
{\color{red}$\circ$} Product  \\
{\color{red}$\circ$} $H_{2}$O, $CO_{2}$, $H^{+}$, N-source, heat  \\
{\color{red}$\circ$} Biomass  \\
{\color{red}$\bullet$} $O_{2}$  \\
\end{frame}


\begin{frame}[shrink] {} 
\color{blue}
    Indicate which statements are correct about the comparison between anaerobic and aerobic processes which make the same product from the same substrate.
 ({\color{green}{Q4b}})\\
\color{black}
\setlength{\parindent}{-0.4cm}
{\color{red}$\circ$} Anaerobic processes are less sustainable  \\
{\color{red}$\bullet$} Anaerobic processes consume less substrate per mol product  \\
{\color{red}$\circ$} Anaerobic processes are more expensive because we need to maintain anaerobic process conditions by keeping $O_{2}$ out.  \\
{\color{red}$\bullet$} Anaerobic processes need less investment because the air compressor is not needed  \\
{\color{red}$\bullet$} Anaerobic processes consume less mechanical energy  \\
\end{frame}


\begin{frame}[shrink] {} 
\color{blue}
    When you choose a production organism the best choice is:
 ({\color{green}{Q4c}})\\
\color{black}
\setlength{\parindent}{-0.4cm}
{\color{red}$\circ$} An organism which can only function under aerobic conditions  \\
{\color{red}$\circ$} An organism which can only function under anaerobic conditions  \\
{\color{red}$\bullet$} An organism which can function both under anaerobic and aerobic conditions  \\
\end{frame}


\subsection{Basics of the black box model}
\begin{frame}[shrink] {} 
\color{blue}
A microorganismis is cultivated under substrate limited conditions in a chemostat at steady state. You want to change the steady state q-rates. How can one achieve this? Choose the correct statements. (Hint: set up the required chemostat balances) ({\color{green}{Q1}})\\
\color{black}
\setlength{\parindent}{-0.4cm}
{\color{red}$\bullet$}  Add extra biomass through the feed, but don’t change the flow rates (in and out) and substrate feed. \\
{\color{red}$\circ$}  Change the airflow. \\
{\color{red}$\circ$}  Change the concentration of trace metals and vitamins in the feed. \\
{\color{red}$\bullet$}  Change the outflow rate \\
{\color{red}$\bullet$}  Change broth volume $V_{L}$ while feed inflow and broth outflow rate remain the same. \\
\end{frame}


\begin{frame}[shrink] {} 
\color{blue}
For a microorganism the following hyperbolic substrate uptake relation is known: ({\color{green}{Q2a}})\\[0.3em]
{\color{gray} $q_s = -0.20\frac{c_s}{2+c_s}$}\\
In this equation, $c_{s}$ is in mol substrate/$m^{3}$ and q-rates are in molar units. What is the value of $q_{s,max}$ and what is the $K_{s}$ value?\\
\color{black}
\setlength{\parindent}{-0.4cm}
{\color{red}$\bullet$}  $q_{s,max}$ is -0.20 and $K_{s}$ is 2 \\
{\color{red}$\circ$}  $q_{s,max}$ is 0.20 and $K_{s}$ is 2 \\
{\color{red}$\circ$}  $q_{s,max}$ is 2 and $K_{s}$ is -0.20 \\
{\color{red}$\circ$}  $q_{s,max}$ is 2 and $K_{s}$ is 0.20 \\
\end{frame}


\begin{frame}[shrink] {} 
\color{blue}
Which is the correct unit for $q_{s,max}$?  ({\color{green}{Q2b}})\\
\color{black}
\setlength{\parindent}{-0.4cm}
{\color{red}$\bullet$}  mol S/mol X / h \\
{\color{red}$\circ$}  mol S/mol X \\
{\color{red}$\circ$}  gram S/mol X / h \\
{\color{red}$\circ$}  gram S/gram X / h \\
\end{frame}


\begin{frame}[shrink] {} 
\color{blue}
What is the correct unit for $K_{s}$ in the given hyperbolic substrate uptake relation (see question 2A) ?  ({\color{green}{Q2c}})\\
\color{black}
\setlength{\parindent}{-0.4cm}
{\color{red}$\circ$}  Kg S/L \\
{\color{red}$\circ$}  g S/L \\
{\color{red}$\circ$}  mol S/L \\
{\color{red}$\bullet$}  mol S/m3 \\
\end{frame}


\begin{frame}[shrink] {} 
\color{blue}
A microorganism consumes a required vitamin and it is known that the
vitamin uptake follows a hyperbolic relation. The affinity parameter for this
hyperbolic vitamin uptake relation equals $K_{vit}$ = 10nmol/L. Furthermore, it
is measured that the vitamin concentration in the cells of this
microorganism is 20nmol/g dry biomass. One wishes to perform a batch
cultivation in a shake flask with a final biomass dry matter concentration of
10g/L. The batch was inoculated with 2g dry biomass/L. The broth
volume in the shake flask does not change.\\[0.3em]
What vitamin concentration is minimally required in the growth medium to
make sure that vitamin is always present in excess ($\frac{c_{vit}}{K_{vit}}>20$) during the
batch cultivation? ({\color{green}{Q3a}})\\
\color{black}
\setlength{\parindent}{-0.4cm}
{\color{red}$\circ$}  200 nmol/L \\
{\color{red}$\bullet$}  360 nmol/L \\
{\color{red}$\circ$}  400 nmol/L \\
{\color{red}$\circ$}  600 nmol/L \\
\end{frame}


\begin{frame}[shrink] {} 
\color{blue}
Why is substrate almost always chosen as the limiting nutrient? (Multiple answers possible) ({\color{green}{Q3b}})\\
\color{black}
\setlength{\parindent}{-0.4cm}
{\color{red}$\bullet$}  Substrate is costly, so you want to use as little of it as possible \\
{\color{red}$\bullet$}  Under substrate excess conditions the organism makes undesired overflow metabolites (e.g. ethanol, acetate) \\
{\color{red}$\circ$}  Substrate always has the highest solubility \\
\end{frame}

\subsection{Energy consuming and energy producing products}
\begin{frame}[shrink] {} 
\color{blue}
The following Herbert-Pirt substrate distribution equation is given (all q-values are not in molar units but are in g of compound i/g dry biomass (X) $h^{-1}$): ({\color{green}{Q1a}})\\[0.3em]
{\color{gray} $q_{s}$=-2.0$\mu$ - 3.0$q_{p}$ - 0.020}\\[0.3em]
What is the unit of $\mu$?\\
\color{black}
\setlength{\parindent}{-0.4cm}
{\color{red}$\circ$}  mol X / h \\
{\color{red}$\circ$}  g X / h \\
{\color{red}$\bullet$}  g X/g X / h \\
\end{frame}


\begin{frame}[shrink] {} 
\color{blue}
What is the unit of $q_{p}$? ({\color{green}{Q1b}})\\
\color{black}
\setlength{\parindent}{-0.4cm}
{\color{red}$\bullet$}  g P / g X / h \\
{\color{red}$\circ$}  mol P / g X / h \\
{\color{red}$\circ$}  g P / g S / h \\
{\color{red}$\circ$}  mol P / mol X / h \\
\end{frame}


\begin{frame}[shrink] {} 
\color{blue}
What is the unit of the parameter with value 0.020? ({\color{green}{Q1c}})\\
\color{black}
\setlength{\parindent}{-0.4cm}
{\color{red}$\circ$}  mol S / mol X / h \\
{\color{red}$\circ$}  g S / mol X /h \\
{\color{red}$\bullet$}  g S / g X / h \\
{\color{red}$\circ$}  g P / g X / h \\
{\color{red}$\circ$}  g X / g X / h \\
\end{frame}


\begin{frame}[shrink] {} 
\color{blue}
What is the unit of the parameter with value 2.0 in the same Herbert-Pirt substrate distribution equation as in question 1A? ({\color{green}{Q1d}})\\
\color{black}
\setlength{\parindent}{-0.4cm}
{\color{red}$\bullet$}  g S / g X \\
{\color{red}$\circ$}  g S \\
{\color{red}$\circ$}  mol S / mol X \\
{\color{red}$\circ$}  g X / mol X \\
\end{frame}


\begin{frame}[shrink] {} 
\color{blue}
And what is the unit of the parameter with value 3.0? ({\color{green}{Q1e}})\\
\color{black}
\setlength{\parindent}{-0.4cm}
{\color{red}$\bullet$}  g S / g P \\
{\color{red}$\circ$}  mol S / mol P \\
{\color{red}$\circ$}  g P / g X \\
\end{frame}


\begin{frame}[shrink] {} 
\color{blue}
The Herbert-Pirt relation of  question 1 ({\color{gray} $q_{s}$=-2.0$\mu$ - 3.0$q_{p}$ - 0.020}) belongs to an organism, that has a $\mu$=0.05 and a $q_{p}$ = 0.05. ({\color{green}{Q2a}})\\[0.3em]
Calculate the value of $q_{s}$.\\
\color{black}
\setlength{\parindent}{-0.4cm}
{\color{red}$\bullet$} -0.27  \\
\end{frame}


\begin{frame}[shrink] {} 
\color{blue}
Assuming the q-values as used and obtained in question 2A, calculate the percentage (\%) of the consumed glucose that is used for biomass formation, product formation and maintenance.  ({\color{green}{Q2b}})\\
\color{black}
\setlength{\parindent}{-0.4cm}
For biomass formation = \underline{\quad } \%\\
For product formation = \underline{\quad } \%\\
For maintenance = \underline{\quad } \%\\[0.5em]
{\color{red}$\bullet$} 37.0; \quad 55.6; \quad 7.4 \\
\end{frame}


\begin{frame}[shrink] {} 
\color{blue}
Another black box model for PDO production is available. The model in this question consists of new equations, all q-rates are now given in mol i / mol X / h, and $c_s$ is in mol S / $m^{3}$. ({\color{green}{Q3a}})\\[0.3em]
\color{gray}
$- q_s = 0.20\frac{{c_s^{}}}{{0.5 + c_s^{}}}$\\[0.3em]
$- q_s = 0.25\mu  + 0.80q_p^{} + 0.005$\\[0.3em]
$q_p(\mu ) = 0.05\frac{\mu }{{0.03 + \mu }}$\\[0.3em]
\color{blue}
Calculate $q_{s,batch}$ and $q_{p,batch}$ of this organism in batch where $c_{s} >>$ 10 mol/$m^{3}$ (=1.8 g/L). Give your answer correct to three decimal places and watch the sign of your answer! What is striking about the obtained value for qs,batch?\\
\color{black}
\setlength{\parindent}{-0.4cm}
$q_{p,batch}$ = \underline{\quad } mol PDO/h/mol x\\
$q_{s,batch}$ = \underline{\quad } molS/h/molX\\[0.5em]
{\color{red}$\bullet$} 0.0477; \quad - 0.200 \\
\end{frame}


\begin{frame}[shrink] {} 
\color{blue}
Calculate the consumed amount of glucose in mol per mol PDO, when the organism is cultivated in batch. Give your answer correct to two decimal places. ({\color{green}{Q3b}})\\
\color{black}
\setlength{\parindent}{-0.4cm}
{\color{red}$\bullet$} 4.19 \\
\end{frame}


\begin{frame}[shrink] {} 
\color{blue}
Calculate which percentage of the consumed glucose is used for growth, product formation and maintenance during batch cultivation.  ({\color{green}{Q3c}})\\
\color{black}
\setlength{\parindent}{-0.4cm}
Growth: \underline{\quad }\%\\
Product formation: \underline{\quad }\%\\
Maintenance: \underline{\quad }\%\\[0.5em]
{\color{red}$\bullet$} 78.4; \quad 19.1; \quad 2.5 \\
\end{frame}


\begin{frame}[shrink] {} 
\color{blue}
 Why is the glucose / PDO ratio in batch much higher (much more unfavourable for product formation) than at an $\mu$$_{opt}$ of 0.0245 $h^{-1}$, as discussed in the lecture of  this unit 3.3? ({\color{green}{Q3d}})\\
\color{black}
\setlength{\parindent}{-0.4cm}
{\color{red}$\circ$}  More glucose is catabolized \\
{\color{red}$\circ$}  More glucose is used for maintenance \\
{\color{red}$\bullet$}  More glucose is used for making biomass, due to the nearly 25 times higher growth rate \\
\end{frame}


\begin{frame}[shrink] {} 
\color{blue}
When is the fraction of substrate consumed for maintenance purposes a major term in the Herbert-Pirt substrate distribution equation? ({\color{green}{Q3e}})\\
\color{black}
\setlength{\parindent}{-0.4cm}
{\color{red}$\circ$}  At high $\mu$ \\
{\color{red}$\bullet$}  At low $\mu$ \\
{\color{red}$\circ$}  Maintenance can always be neglected. \\
\end{frame}


\begin{frame}[shrink] {} 
\color{blue}
A PDO producing mutant microorganism is available. In this mutant organism, qp has been increased by increasing enzyme levels in the PDO pathway. Which parameters of the black box model have very likely been changed? ({\color{green}{Q4a}})\\
\color{black}
\setlength{\parindent}{-0.4cm}
{\color{red}$\circ$}  $q_{s,max}i$ and $K_s$ \\
{\color{red}$\circ$}  a, b, and $m_s$ \\
{\color{red}$\bullet$}  $\alpha$ and $\beta$ \\
\end{frame}


\begin{frame}[shrink] {} 
\color{blue}
The PDO black box model for a mutant microorganism shows a new $q_{p}$($\mu$)-relation: ({\color{green}{Q4b}})\\[0.3em]
{\color{gray} $q_p = 0.10 \frac{\mu}{0.030+\mu}$}\\[0.3em]
The $\mu$$_{max}$ was calculated to be 0.48 $h^{-1}$ (you can check this yourself). 
Why is the $\mu$$_{max}$ of the mutant microorganism lower than that for the wild type?\\
\color{black}
\setlength{\parindent}{-0.4cm}
{\color{red}$\circ$}  PDO inhibits growth \\
{\color{red}$\circ$}  A lower maximum growth rate is an unfortunate side effect of the genetic changes \\
{\color{red}$\bullet$}  At the same $q_{s}$ (=$q_{s,max}$ in batch) more substrate goes to PDO resulting in less substrate used for growth, leading to a lower $\mu_{max}$ \\
\end{frame}


\begin{frame}[shrink] {} 
\color{blue}
Economically, the mol S / mol P ratio should be as low as possible. The
black box model shows that this ratio is only a function of $\mu$. It can be
determined that
\color{gray}
$|\frac{q_s}{q_p}| = p_1+\frac{p_2}{\mu}+p_3 \cdot{} \mu$ ({\color{green}{Q4c}})\\[0.3em]
\color{blue}
Calculate the parameter values $p_{1}$, $p_{2}$ and $p_{3}$ for the black box model of the mutant microorganism. Assume that the Herbert-Pirt equation has not changed and is still equal to \\
\color{gray}
$q_{s}$ = 0.25$\mu$ + 0.8$q_{p}$ + 0.005. \\
\color{black}
\setlength{\parindent}{-0.4cm}
p1 = \underline{\quad }\\
p2 = \underline{\quad }\\
p3 = \underline{\quad }\\[0.5em]
{\color{red}$\bullet$} 0.925 \quad 0.0015 \quad 2.5\\
\end{frame}


\begin{frame}[shrink] {} 
\color{blue}
    Calculate the $\mu$-value for which the mol S / mol P is minimal. Give your answer correct to three decimal places ({\color{green}{Q4d}})\\
$\mu$$_{opt}$ = \underline{\quad } $h^{-1}$\\
\color{black}
\setlength{\parindent}{-0.4cm}
{\color{red}$\bullet$}  \quad 0.0245\\
\end{frame}


\begin{frame}[shrink] {} 
\color{blue}
    Calculate $q_{s,opt}$ and $q_{p,opt}$. ({\color{green}{Q4e}})\\
\color{black}
\setlength{\parindent}{-0.4cm}
$q_{s,opt}$ = \underline{\quad } mol S / mol X / h\\
$q_{p,opt}$ =\underline{\quad } mol P / mol X / h\\[0.5em]
{\color{red}$\bullet$}   - 0.0471; \quad 0.0450\\
\end{frame}


\begin{frame}[shrink] {} 
\color{blue}
    Calculate the mol S / mol PDO ratio at $\mu$$_{opt}$.Give your answer correct to three decimal places. ({\color{green}{Q4f}})\\
$\frac{mol S}{mol PDO}$\\
\color{black}
\setlength{\parindent}{-0.4cm}
{\color{red}$\bullet$}  1.047
\end{frame}


\begin{frame}[shrink] {} 
\color{blue}
    When mutant microorganisms with an even higher $q_{p}$ are obtained, the $q_{s}$/$q_{p}$ ratio decreases asymptotically to a value of  ({\color{green}{Q4g}})\\
\color{black}
\setlength{\parindent}{-0.4cm}
{\color{red}$\circ$}  0.25 mol glucose/mol PDO \\
{\color{red}$\bullet$}  0.80 mol glucose/mol PDO \\
{\color{red}$\circ$}  0.005 mol glucose/mol PDO \\
\end{frame}
\subsection{A PDO black box model: experiments for parameter identification}
\begin{frame}[shrink] {} 
\color{blue}
If you know that the production of a product from substrate leads to energy production for the cell, what type of process would you choose? ({\color{green}{Q1}})\\
\color{black}
\setlength{\parindent}{-0.4cm}
{\color{red}$\bullet$}  Anaerobic \\
{\color{red}$\circ$}  Aerobic \\
\end{frame}


\begin{frame}[shrink] {} 
\color{blue}
The substrate is transported into the cell by a  transport protein.  ({\color{green}{Q2a}})\\
Choose the correct type of substrate uptake ($q_{s}$) relation in the Black Box model\\
\color{black}
\setlength{\parindent}{-0.4cm}
{\color{red}$\bullet$}  Hyperbolic in $c_{s}$ \\
{\color{red}$\circ$}  Exponential in $c_{s}$ \\
{\color{red}$\circ$}  Linear in $c_{s}$ \\
\end{frame}


\begin{frame}[shrink] {} 
\color{blue}
For the same Black Box model as mentioned in question 2, choose the number of terms that appear in the Herbert-Pirt relation in this case. ({\color{green}{Q2b}})\\
\color{black}
\setlength{\parindent}{-0.4cm}
{\color{red}$\circ$}  Herbert-Pirt relation with 3 terms \\
{\color{red}$\bullet$}  Herbert-Pirt relation with 2 terms \\
{\color{red}$\circ$}  Herbert-Pirt relation with 1 term \\
\end{frame}


\begin{frame}[shrink] {} 
\color{blue}
In case of an energy producing product, how does the  $q_{p}$($\mu$)-function look like for the black box model? In this case, $q_{p}$ is: ({\color{green}{Q2c}})\\
\color{black}
\setlength{\parindent}{-0.4cm}
{\color{red}$\circ$}  non-linear in $\mu$ \\
{\color{red}$\bullet$}  linear in $\mu$ \\
{\color{red}$\circ$}  proportional in $\mu$ \\
\end{frame}


\begin{frame}[shrink] {} 
\color{blue}
To set up a black box model for an energy generating product under anaerobic conditions, chemostat experiments must be performed to obtain the parameter values. ({\color{green}{Q3}})\\[0.3em] 
How many different chemostat experiments are minimally required? \\
\color{black}
\setlength{\parindent}{-0.4cm}
{\color{red}$\circ$}  1 \\
{\color{red}$\bullet$}  2 \\
{\color{red}$\circ$}  3 \\
{\color{red}$\circ$}  4 \\
\end{frame}


\begin{frame}[shrink] {} 
\color{blue}
From an organism that does not produce any product (only biomass), the following dataset is available from chemostat experiments: ({\color{green}{Q4a}})\\
\color{gray}
\begin{tabular}[ ]{l l l l} 
%    Experiment \#  & 2 & 3 & 4 \\
Experiment\# & $\mu$  & -$q_{s}$ & $c_{s}$   \\
1 & 0.007 & 0.024 & 1.0  \\
2 & 0.018 & 0.045 & 2.0  \\
3 & 0.037 & 0.083 & 4.0  \\
4 & 0.125 & 0.250 & 20.0  \\
5 & 0.208 & 0.417 & 100.0  \\
\end{tabular}  \\
\color{blue}
\footnotesize $\mu$: in h-1;  -$q_{s}$: in g glucose/gDW* h-1;  $c_{s}$: in mg glucose/L; *gDW = gram dry weight biomass\\
%
%\begin{figure}[ ]
%\begin{minipage}{0.48\linewidth}
%\centerline{\includegraphics[width=4.0cm,high=5.0cm]{html/3_4_files/figure1.jpg}}
%\caption{$q_{s}$ is plotted over $\mu$}
%%Figure 1: $q_{s}$ is plotted over $\mu$.\\
%\end{minipage}
%\hfill
%\begin{minipage}{0.48\linewidth}
%\centerline{\includegraphics[width=4.0cm,high=5.0cm]{html/3_4_files/figure2.jpg}}
%\caption{$q_{s}$ is plotted against $c_{s}$}
%\end{minipage}
%\end{figure}
%%From this dataset, two plots have been made:\\
%%Figure 2: $q_{s}$ is plotted against $c_{s}$\\
%\end{frame}
%
%\begin{frame}[shrink] {} 
%\color{blue}
\normalsize Which equations are relevant for the black box model of this organism?\\
\color{black}
\setlength{\parindent}{-0.4cm}
{\color{red}$\bullet$}  hyperbolic substrate uptake rate equation \\
{\color{red}$\bullet$}  Herbert-Pirt substrate distribution relation \\
{\color{red}$\circ$}  $q_{p}$($\mu$) relation \\
\end{frame}


\begin{frame}[shrink] {} 
\color{blue}
The values for some of the Black Box model parameters can be determined using the figures of question 4. ({\color{green}{Q4b}})\\
Herbert Pirt substrate distribution relation:\\[0.3em]
\color{gray}
$−q_s=a\mu+m_s$ \\[0.3em]
\color{blue}
Hyperbolic substrate uptake rate:\\
\color{gray}
$- q_s^{} = q_{s,\max }^{}\frac{{c_s^{}}}{{K_s^{} + c_s^{}}}$\\[0.3em]
\color{black}
\setlength{\parindent}{-0.4cm}
{\color{red}$\circ$} $a$ =  \underline{\quad} g S/g X \\
{\color{red}$\circ$} $m_s$ = \underline{\quad} g S/g X $h^{-1}$\\
{\color{red}$\circ$} $K_s$ = \underline{\quad} mg substrate/L \\
{\color{red}$\circ$} $-q_{s,max}$ =\underline{\quad} g S/g X\\[0.5em]
{\color{red}$\bullet$} 1.95; \quad 0.010; \quad 20; \quad 0.50
\end{frame}


\begin{frame}[shrink] {} 
\color{blue}
How many  experiments are minimally required to determine all parameters for the black box model equations below? ({\color{green}{Q4c}})\\
\color{gray}
$- q_s^{} = a\,\mu  + m_s^{}$\\[0.3em]
$- q_s^{} = q_{s,\max }^{}\frac{{c_s^{}}}{{K_s^{} + c_s^{}}}$\\[0.3em]
\color{blue}
Number of experiments = \underline{\quad }\\[0.5em]
\color{black}
\setlength{\parindent}{-0.4cm}
{\color{red}$\bullet$} 2\\
\end{frame}


\begin{frame}[shrink] {} 
\color{blue}
One wishes to produce the organism described in question 4 in a batch fermentation, which is inoculated with 10 kg biomass. Calculate $\mu_{max}$. ({\color{green}{Q5a}})\\
\color{black}
\setlength{\parindent}{-0.4cm}
$\mu_{max} =\underline{\quad } h^{-1}$\\[0.5em]
{\color{red}$\bullet$} 0.245\\
\end{frame}


\begin{frame}[shrink] {} 
\color{blue}
    How much glucose (in kg) is needed at the start of the batch to achieve 100 kg biomass at the end of the batch fermentation? Assume that all glucose is consumed at the end of the batch fermentation.  ({\color{green}{Q5b}})\\
\color{black}
\setlength{\parindent}{-0.4cm}
Glucose: \underline{\quad } kg at the start of the batch.\\[0.5em]
{\color{red}$\bullet$} 183.7\\
\end{frame}


\begin{frame}[shrink] {} 
\color{blue}
    How much of this glucose (in kg) is used for maintenance in batch? ({\color{green}{Q5c}})\\
\color{black}
\setlength{\parindent}{-0.4cm}
Glucose for maintenance in batch: \underline{\quad } kg\\[0.5em]
{\color{red}$\bullet$} 3.67\\
\end{frame}


\begin{frame}[shrink] {} 
\color{blue}
    How many hours does the batch take? ({\color{green}{Q5d}})\\
\color{black}
\setlength{\parindent}{-0.4cm}
Duration:\underline{\quad } hours\\[0.5em]
{\color{red}$\bullet$}9.4\\
\end{frame}
\subsection{Black box models: The PDO process reaction as function of $\mu$}
\subsection{PDO continuous process design: calculation of inputs and outputs using the process reaction}
\subsection{Aerobic PDO process: improving sustainability}
\subsection{guest lecture balances}

\section{Fermentation Design}
\section{Up- and Downstream Process Integration}
\section{Process Evaluation and Sustainability}
\section{Closing lecture and survey}
\end{document}
