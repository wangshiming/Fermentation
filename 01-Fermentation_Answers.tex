
\documentclass[]{beamer}
\usetheme{Goettingen}
\usepackage{amsmath,siunitx,comment,CJK}
\begin{document}
\begin{CJK}{UTF8}{gbsn}
\parskip=1cm plus 1pt
\newcounter{answers}
\section{Biotechnology for Biobased Products}
\subsection{ Why develop the biobased economy?}
\setcounter{answers}{0}


\begin{frame}[shrink] {} 
\addtocounter{answers}{1}
\color{blue}
  Please indicate the drivers that promote the transition to a biobased economy
 ({\color{green}{Q\arabic{answers}}})\\
\color{black}
\setlength{\parindent}{-0.4cm}
{\color{red}$\circ$} Increasing antibiotic resistance  \\
{\color{red}$\bullet$} Concerns about climate change  \\
{\color{red}$\circ$} Novel nanotechnology   \\
{\color{red}$\bullet$} Energy-security of nations  \\
{\color{red}$\bullet$} Lowering waste production (incl. $CO_{2})$   \\
{\color{red}$\circ$} Development of better storage options for electrical energy  \\
{\color{red}$\bullet$} Depletion of fossil resources   \\
{\color{red}$\circ$} Obamacare  \\
{\color{red}$\circ$} Increasing use of wind energy  \\
{\color{red}$\circ$} Increasing availability of birth control methods  \\
{\color{red}$\bullet$} The need for sustainable organic (carbon-containing) molecules  \\
{\color{red}$\bullet$} A growing world population that deserves access to basic needs and a good environment.  \\
{\color{red}$\bullet$} The possibility to add sustainable value to the food chain (integral use of biomass)  \\
{\color{red}$\circ$} Increasing feasibility of 3D printing technology  \\
{\color{red}$\circ$} Decreasing use of solar energy  \\
{\color{red}$\bullet$} Opportunities for local farmers  \\
{\color{red}$\circ$} Decreasing population growth  \\
\end{frame}


\begin{frame}[shrink] {} 
\addtocounter{answers}{1}
\color{blue}
  Does biobased production compete with food supply?
 ({\color{green}{Q\arabic{answers}}})\\
\color{black}
\setlength{\parindent}{-0.4cm}
{\color{red}$\circ$} No, never. Biobased products are always made from waste streams.  \\
{\color{red}$\bullet$} Yes, sometimes. Although it is desired to use only waste streams as a feedstock, the current state-of-the art is to also use food crops.  \\
{\color{red}$\circ$} Yes, always. Biobased products are by definition made from food crops.  \\
\end{frame}


\begin{frame}[shrink] {} 
\addtocounter{answers}{1}
\color{blue}
  Is biobased the same as biodegradable?
 ({\color{green}{Q\arabic{answers}}})\\
\color{black}
\setlength{\parindent}{-0.4cm}
{\color{red}$\circ$} Yes, biobased products are always biodegradable.  \\
{\color{red}$\circ$} No, biodegradable products are always biobased but not vice versa.  \\
{\color{red}$\bullet$} No, only some biobased products are biodegradable.  \\
{\color{red}$\circ$} No, biobased products are never biodegradable.  \\
\end{frame}


\begin{frame}[shrink] {} 
\addtocounter{answers}{1}
\color{blue}
  Why is the necessity for biobased production higher for chemicals than for fuels? 
 ({\color{green}{Q\arabic{answers}}})\\
\color{black}
\setlength{\parindent}{-0.4cm}
{\color{red}$\bullet$} Because the carbon skeleton of molecules cannot be produced by solar or wind energy.   \\
{\color{red}$\circ$} Because there are more chemicals than fuels in the world.  \\
{\color{red}$\circ$} In the future, fuels will no longer be necessary.   \\
\end{frame}


\subsection{ Industrial Biotechnology}
\setcounter{answers}{0}


\begin{frame}[shrink] {} 
\addtocounter{answers}{1}
\color{blue}
  Which three biological systems are most commonly used in industrial biotechnology for the production of bulk or fine chemicals? 
 ({\color{green}{Q\arabic{answers}}})\\
\color{black}
\setlength{\parindent}{-0.4cm}
{\color{red}$\bullet$} Bacteria   \\
{\color{red}$\circ$} Archaea   \\
{\color{red}$\circ$} Protozoa  \\
{\color{red}$\bullet$} Fungi  \\
{\color{red}$\bullet$} Enzymes  \\
{\color{red}$\circ$} Amoebae  \\
{\color{red}$\circ$} Mammalian cells  \\
{\color{red}$\circ$} Viruses  \\
\end{frame}


\begin{frame}[shrink] {} 
\addtocounter{answers}{1}
\color{blue}
  What is the difference between ‘traditional’ and ‘industrial’ biotechnology?
 ({\color{green}{Q\arabic{answers}}})\\
\color{black}
\setlength{\parindent}{-0.4cm}
{\color{red}$\circ$} Traditional biotechnology uses traditional, folkloristic techniques whereas industrial biotechnology is done in large industrial factories.  \\
{\color{red}$\circ$} Traditional biotechnology is always done with microorganisms found in nature, whereas industrial biotechnology always uses microorganisms from a laboratory.  \\
{\color{red}$\bullet$} Traditional biotechnology  is used to make traditional products such as bread and alcoholic beverages whereas industrial biotechnology focusses on the production of bulk and certain specific chemicals.  \\
\end{frame}


\begin{frame}[shrink] {} 
\addtocounter{answers}{1}
\color{blue}
  Which five of the following products are generally classified as industrial biotechnology?
 ({\color{green}{Q\arabic{answers}}})\\
\color{black}
\setlength{\parindent}{-0.4cm}
{\color{red}$\circ$} Alcoholic beverages  \\
{\color{red}$\bullet$} Biopharmaceuticals  \\
{\color{red}$\bullet$} Enzymes   \\
{\color{red}$\bullet$} Biofuels  \\
{\color{red}$\circ$} Fossil fuels  \\
{\color{red}$\bullet$} Bioplastics  \\
{\color{red}$\circ$} Bread  \\
{\color{red}$\circ$} Quorn  \\
{\color{red}$\circ$} Petrochemical plastics  \\
{\color{red}$\bullet$} Biobased chemicals  \\
{\color{red}$\circ$} Blue cheese  \\
\end{frame}


\begin{frame}[shrink] {} 
\addtocounter{answers}{1}
\color{blue}
  Please mark the statements that are correct: 
 ({\color{green}{Q\arabic{answers}}})\\
\color{black}
\setlength{\parindent}{-0.4cm}
{\color{red}$\circ$} All microorganisms cause diseases.  \\
{\color{red}$\circ$} Enzymes are the smallest microorganisms known.   \\
{\color{red}$\bullet$} Microorganisms can self-replicate.  \\
{\color{red}$\bullet$} Microorganisms are found pretty much everywhere.  \\
{\color{red}$\circ$} Bacteria and fungi are the only types of microorganisms.  \\
{\color{red}$\bullet$} (Micro)organisms can be divided in 3 domains: bacteria, archaea and eukaryotes.  \\
{\color{red}$\circ$} Microorganisms are usually 1 mm in size.  \\
\end{frame}


\begin{frame}[shrink] {} 
\addtocounter{answers}{1}
\color{blue}
  What is fermentation? 
 ({\color{green}{Q\arabic{answers}}})\\
\color{black}
\setlength{\parindent}{-0.4cm}
{\color{red}$\circ$} The process where enzymes convert organic substrates to useful products. This process is either aerobic or anaerobic.  \\
{\color{red}$\bullet$} The conversion of organic substrates by microorganisms. In the classical definition, this process is anaerobic but the definition is also used wider, including aerobic processes.  \\
{\color{red}$\circ$} The anaerobic conversion of fossil feedstocks by microorganisms.  \\
{\color{red}$\circ$} The conversion of fossil feedstocks by enzymes. In the classical definition, this process is anaerobic but the definition is also used wider, including aerobic processes.  \\

\end{frame}
\subsection{Feedstocks and biobased products }
\setcounter{answers}{0}


\begin{frame}[shrink] {} 
\addtocounter{answers}{1}
\color{blue}
  What is an important disadvantage of ‘first generation’ biomass?
 ({\color{green}{Q\arabic{answers}}})\\
\color{black}
\setlength{\parindent}{-0.4cm}
{\color{red}$\bullet$} Competition with food and feed production.  \\
{\color{red}$\circ$} The sugars in these feedstocks are not readily available for fermentation.  \\
{\color{red}$\circ$} These feedstocks are often produced in remote locations leading to high transportation costs.  \\
\end{frame}


\begin{frame}[shrink] {} 
\addtocounter{answers}{1}
\color{blue}
  What is ‘second generation’ biomass?
 ({\color{green}{Q\arabic{answers}}})\\
\color{black}
\setlength{\parindent}{-0.4cm}
{\color{red}$\circ$} Residues from fossil feedstocks that can be fermented by microorganisms.  \\
{\color{red}$\bullet$} Plant materials that are not used for food and feed production.  \\
{\color{red}$\circ$} All plant materials that are available for fermentation.  \\
{\color{red}$\circ$} The microbial biomass that was formed during the fermentation process.  \\
\end{frame}


\begin{frame}[shrink] {} 
\addtocounter{answers}{1}
\color{blue}
  What could be potential ‘third generation’ biomass?
 ({\color{green}{Q\arabic{answers}}})\\
\color{black}
\setlength{\parindent}{-0.4cm}
{\color{red}$\circ$} Corn  \\
{\color{red}$\bullet$} Photosynthetic algae  \\
{\color{red}$\circ$} Switchgrass  \\
{\color{red}$\circ$} Municipal waste  \\
{\color{red}$\circ$} Sugar cane  \\
\end{frame}


\begin{frame}[shrink] {} 
\addtocounter{answers}{1}
\color{blue}
  Which factors should be taken into account when selecting a feedstock for biobased production?
 ({\color{green}{Q\arabic{answers}}})\\
\color{black}
\setlength{\parindent}{-0.4cm}
{\color{red}$\circ$} Taste  \\
{\color{red}$\circ$} Genetic modification potential of the microorganism  \\
{\color{red}$\bullet$} Biochemical composition  \\
{\color{red}$\bullet$} Yield  \\
{\color{red}$\circ$} Olfactory qualities  \\
{\color{red}$\bullet$} Sustainability  \\
\end{frame}


\begin{frame}[shrink] {} 
\addtocounter{answers}{1}
\color{blue}
  What is the main difference in biochemical composition between first-generation and second-generation lignocellulosic feedstocks? 
 ({\color{green}{Q\arabic{answers}}})\\
\color{black}
\setlength{\parindent}{-0.4cm}
{\color{red}$\bullet$} First-generation feedstocks contain sugars mainly in the form of starches, whereas lignocellulosic feedstocks contain also other polymers like cellulose, hemicellulose and lignin.   \\
{\color{red}$\circ$} Second-generation feedstocks contain much more sugar than first-generation feedstocks.   \\
{\color{red}$\circ$} Second-generation lignocellulosic feedstocks have a more simple composition than first-generation feedstocks since they don't contain polymers.   \\
\end{frame}


\begin{frame}[shrink] {} 
\addtocounter{answers}{1}
\color{blue}
Why is, the difference in biochemical composition between first-generation and second-generation feedstocks important for biobased production processes?
 ({\color{green}{Q\arabic{answers}}})\\
(Please note, multiple answers possible!)
  \\
\color{black}
\setlength{\parindent}{-0.4cm}
{\color{red}$\bullet$} The biochemical composition has consequences for the pretreatment  of biomass.  \\
{\color{red}$\circ$} The biochemical composition has a large influence on the taste and olfactory qualities of the crop.  \\
{\color{red}$\circ$} The downstream processing of the product will be complicated.  \\
{\color{red}$\bullet$} The biochemical composition is reflected in the yield of fermentable sugar on biomass.  \\
\end{frame}


\begin{frame}[shrink] {} 
\addtocounter{answers}{1}
\color{blue}
  Which two polymers can -after depolymerisation/hydrolyzation- be used in fermentation? 
 ({\color{green}{Q\arabic{answers}}})\\
\color{black}
\setlength{\parindent}{-0.4cm}
{\color{red}$\bullet$} Cellulose  \\
{\color{red}$\bullet$} Hemicellulose  \\
{\color{red}$\circ$} Lignin  \\
\end{frame}


\begin{frame}[shrink] {} 
\addtocounter{answers}{1}
\color{blue}
  What is the sequence of steps in pretreatment of second-generation biomass? 
 ({\color{green}{Q\arabic{answers}}})\\
\color{black}
\setlength{\parindent}{-0.4cm}
{\color{red}$\circ$}    \\
Lignin removal\\
Partial enzymatic hydrolysis\\
Final chemical hydrolysis\\
{\color{red}$\circ$}    \\
Opening the fibre structure by enzymatic treatment \\
Lignin removal\\
Depolymerisation by mechanical and/or chemical treatment\\
{\color{red}$\bullet$}   \\
Opening the fibre structure by mechanical and/or chemical pretreatment\\
Lignin removal\\
Depolymerisation by enzymatic treatment\\
\end{frame}

\subsection{Process to bio-PDO}
\setcounter{answers}{0}


\begin{frame}[shrink] {} 
\addtocounter{answers}{1}
\color{blue}
  Is PDO a biobased or biodegradable product?
 ({\color{green}{Q\arabic{answers}}})\\
\color{black}
\setlength{\parindent}{-0.4cm}
{\color{red}$\bullet$} It is biodegradable, and can be produced via either petrochemical or biobased production.  \\
{\color{red}$\circ$} It can be produced via either petrochemical or biobased production, but it is not biodegradable.   \\
{\color{red}$\circ$} None, although PDO can be produced via biobased production.  \\
\end{frame}


\begin{frame}[shrink] {} 
\addtocounter{answers}{1}
\color{blue}
  Is PTT a biobased product?
 ({\color{green}{Q\arabic{answers}}})\\
\color{black}
\setlength{\parindent}{-0.4cm}
{\color{red}$\circ$} Yes, both monomers are produced via a biobased route.  \\
{\color{red}$\circ$} Yes, although the PDO fraction of the polymer can also be produced via petrochemistry.   \\
{\color{red}$\circ$} No, both monomers can only be produced petrochemically.   \\
{\color{red}$\bullet$} No, although the PDO fraction of the polymer can be produced from biomass, the terephthalate fraction is made from oil.   \\
\end{frame}


\begin{frame}[shrink] {} 
\addtocounter{answers}{1}
\color{blue}
  What is a ‘platform chemical’?
 ({\color{green}{Q\arabic{answers}}})\\
\color{black}
\setlength{\parindent}{-0.4cm}
{\color{red}$\circ$} A molecule that you can use to make offshore platforms.  \\
{\color{red}$\circ$} A monomer molecule that can be used to make a polymer.   \\
{\color{red}$\bullet$} A molecule that can be further processed for various applications.  \\
\end{frame}


\begin{frame}[shrink] {} 
\addtocounter{answers}{1}
\color{blue}
  What are the benefits of biobased production versus petrochemical production of PDO?
 ({\color{green}{Q\arabic{answers}}})\\
\color{black}
\setlength{\parindent}{-0.4cm}
{\color{red}$\bullet$} Lower capital investments  \\
{\color{red}$\circ$} Higher quality PDO   \\
{\color{red}$\bullet$} Lower prices and higher availability  \\
{\color{red}$\bullet$} Less energy consumption  \\
{\color{red}$\circ$} A change from a straight polymer to a curved polymer leading to better polymer properties  \\
{\color{red}$\circ$} Bio-PDO has beneficial chemical properties compared to petrochemically produced PDO  \\
{\color{red}$\bullet$} Reduction of GHG emissions  \\
{\color{red}$\bullet$} Lower manufacturing costs  \\
\end{frame}


\begin{frame}[shrink] {} 
\addtocounter{answers}{1}
\color{blue}
  What are the four main benefits of PTT versus PET?
 ({\color{green}{Q\arabic{answers}}})\\
\color{black}
\setlength{\parindent}{-0.4cm}
{\color{red}$\bullet$} Better polymerisation   \\
{\color{red}$\circ$} Lower capital investments  \\
{\color{red}$\bullet$} Less energy consumption  \\
{\color{red}$\bullet$} A change from a straight polymer to a curved polymer leading to better polymer properties  \\
{\color{red}$\circ$} Reduction of GHG emissions  \\
{\color{red}$\circ$} Lower manufacturing costs  \\
{\color{red}$\bullet$} No heavy metals, resulting in easier recycling  \\

\end{frame}


\subsection{The benefits for society}
\setcounter{answers}{0}


\begin{frame}[shrink] {} 
\addtocounter{answers}{1}
\color{blue}
  What is the driver that will ultimately determine if companies will implement a biobased process?
 ({\color{green}{Q\arabic{answers}}})\\
\color{black}
\setlength{\parindent}{-0.4cm}
{\color{red}$\bullet$} Economic viability   \\
{\color{red}$\circ$} Public perception  \\
{\color{red}$\circ$} Environmental friendliness of the process   \\
{\color{red}$\circ$} Governmental lobby  \\
\end{frame}


\begin{frame}[shrink] {} 
\addtocounter{answers}{1}
\color{blue}
  Why can a biobased economy improve the global fuel security?
 ({\color{green}{Q\arabic{answers}}})\\
\color{black}
\setlength{\parindent}{-0.4cm}
{\color{red}$\circ$} Oil-producing countries will be able to produce oil from biomass as well, increasing their total oil output.  \\
{\color{red}$\circ$} Biobased energy sources contain more energy than oil, reducing global demand.  \\
{\color{red}$\bullet$} Biomass required for biobased energy sources is not limited to a few geographic regions, which is the case for fossil fuels.  \\
{\color{red}$\circ$} Biomass can be used to produce fossil fuels.   \\
\end{frame}


\begin{frame}[shrink] {} 
\addtocounter{answers}{1}
\color{blue}
  What are Capital Expenditures (CAPEX)?
 ({\color{green}{Q\arabic{answers}}})\\
\color{black}
\setlength{\parindent}{-0.4cm}
{\color{red}$\bullet$} The costs associated with setting up the process facility  \\
{\color{red}$\circ$} The costs associated with running the facility  \\
{\color{red}$\circ$} The costs associated with taxes and legislation  \\
{\color{red}$\circ$} The profit margin on the product  \\
\end{frame}


\begin{frame}[shrink] {} 
\addtocounter{answers}{1}
\color{blue}
  What are Operational Expenditures (OPEX)?
 ({\color{green}{Q\arabic{answers}}})\\
\color{black}
\setlength{\parindent}{-0.4cm}
{\color{red}$\circ$} The costs associated with setting up the process facility  \\
{\color{red}$\bullet$} The costs associated with running the facility  \\
{\color{red}$\circ$} The costs associated with taxes and legislation  \\
{\color{red}$\circ$} The profit margin on the product  \\
\end{frame}


\begin{frame}[shrink] {} 
\addtocounter{answers}{1}
\color{blue}
  Which factors should be included to assess social sustainability? 
 ({\color{green}{Q\arabic{answers}}})\\
\color{black}
\setlength{\parindent}{-0.4cm}
{\color{red}$\bullet$} Employment  \\
{\color{red}$\circ$} GHG emissions  \\
{\color{red}$\circ$} Profitability  \\
{\color{red}$\bullet$} Food security  \\
{\color{red}$\circ$} Tax regulations  \\
{\color{red}$\bullet$} Social wellbeing  \\
{\color{red}$\circ$} CAPEX and OPEX  \\
{\color{red}$\bullet$} Use of socially acceptable technology  \\
{\color{red}$\bullet$} Equal distribution of benefits  \\
\end{frame}


\begin{frame}[shrink] {} 
\addtocounter{answers}{1}
\color{blue}
  What is a Life Cycle Assessment?
 ({\color{green}{Q\arabic{answers}}})\\
\color{black}
\setlength{\parindent}{-0.4cm}
{\color{red}$\circ$} A complete assessment of the process to determine the recyclability of the product.  \\
{\color{red}$\bullet$} A complete assessment of the process to determine the (potential) environmental impact.  \\
{\color{red}$\circ$} A complete assessment of the process to determine the costs of the process.  \\
{\color{red}$\circ$} A marketing campaign to promote the positive influence of a product on the life cycle of humans.  \\

\end{frame}
\section{Balances and Microbial Rates}
\section{The Black Box Model and Process Reaction}
\section{Fermentation Design}
\section{Up- and Downstream Process Integration}
\section{Process Evaluation and Sustainability}
\section{Closing lecture and survey}

\end{CJK}
\end{document}
